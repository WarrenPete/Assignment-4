\documentclass[]{article}
\usepackage[utf8]{inputenc}
\usepackage{amsmath}
\usepackage{amsfonts}
\usepackage{amssymb}
\author{KAHIGIRIZA PETER WARREN}
\title{LITERATURE REVIEW ON WAZE APP.}
\begin{document}

\maketitle
Waze is a navigation and mapping app that operates based on community contributions. The user inputs in their desired location in the app's search bar and Waze will compute the shortest route in terms of time or distance, depending on the user's choice.\\
 What's interesting about Waze is its crowd-sourcing aspect. Waze relies on data from its large pool of users to build its network of maps. \\
 By tracing the users' journey and keeping track of congested routes, the app is able to provide immediate solutions for each user, constantly updating the user on the best route available and giving the option to reroute at times. \\
 Through notifications provided by the community on tolls, heavy traffic, road blocks, bad weather and construction work, among other traffic information, users are able to gain some foresight on their journey and plan their route accordingly. The user may also set preferences on their profile such that they may avoid roads with unfavorable conditions, such as dirt paths. \\
 Besides being an app for navigation, it is also an app that is tailored to its user (a driver) focusing on their intricate needs while on the road. For instance, a driver who needs to visit the nearest fuel station but also sticks with a particular brand of fuel because she has a loyalty card with them. By declaring those preferences, the app can cater to the user's desires.\cite{r1} 
\\

Strengths\\

 • Provides live data and updates on traffic and ETA, courtesy of its large and active user base.\\
 
 • Huge databases are stored on the server, not required to be downloaded onto the phone. \\
 
• Option to sync with Calendar and Facebook Events with corresponding locations. \\

• Option to sync with social networks – the user is able to see whether their friends are online and where they've recently checked in. \\

• Has a platform for messaging other users, either for small talk or asking about a situation on the road (e.g. cause of congestion, road rallies).\\

 • Able to add home/favorite locations.\\
 
 • Has a variety of services for the user (e.g. nearby fuel stations, parking lots, rest areas, updates on fuel prices). \\
 
• Good customization – able to set preferences for brand and type of fuel, choosing roads with toll, avoiding dirt roads etc.\\

 • Certain alerts and ads do not pop up to distract the user unless the user had stopped driving for at least 10 seconds. \\
  \\
  \\
  
Weaknesses \\

• Unreliable network connection. Sometimes the GPS signal and connection to server is lost while navigating. \\

• Issues with app crashing – need to restart the phone for the app to work, having to log in again and losing Waze history in the process. \\

• Relying on user contribution is an advantage but can also be a crutch – if there is not a lot of user activity in an area then the traffic and map information provided will not be as accurate and effective. On the other hand, having too many users will result in many icons cluttering up the phone screen. \\

• Traffic information pop-ups can be distracting. \\

• Consumes a lot of data and battery as Waze collects information even when the user is not navigating.\cite{r2}   \\
 
\bibliographystyle{unsrt} 
\bibliography{References}
\end{document}